\section{Ievads}

\subsection{Definīcijas un saīsinājumi}

\begin{tabular}{ |c|wc{12cm}| }
    \hline
    \rowcolor{blue!30}
    \textbf{Saīsinājums} & \textbf{Skaidrojums}             \\
    \hline
    RTU                  & Rīgas Tehniskā Universitāte      \\
    \hline
    BDAS                 & Bakalaura darbu atbalsta sistēma \\
    \hline
\end{tabular}

\subsection{Dokumenta nolūks}

Šis dokuments apraksta studiju projekta studentu zināšanas vērtēšanas sistēmu kursam "Risinājumu
algoritmizācija un programmēšana" (tālāk tekstā Scoring) programmatūras prasības.

Dokuments ir paredzēts studiju projekta ietvaros izstrādātās programmatūras izstrādē, ieviešanā un
uzturēšanā iesaistītajām pusēm:

\begin{itemize}[noitemsep, nolistsep]
    \item pasūtītāja (RTU) pasniedzēji, kuri atbildīgi par projekta nodevumu pieņemšanu un izvērtēšanu
    \item izstrādātāja tehniskie speciālisti, kuri atbildīgi par tā realizāciju – projektēšanu un implementēšanu.
\end{itemize}

\subsection{Darbības sfēra}

Studiju projekta ietvaros tiek izstrādāta sistēma, kas ļaus RTU pasniedzējiem izveidot pārbaudes
testus un definēt to vērtēšanas kritērijus un studentiem veikt šo testu izpildi.

Šis dokuments apraksta programmatūru, kas ļaus:

\begin{itemize}[noitemsep, nolistsep]
    \item Uzglabāt informāciju par sistēmas lietotājiem;
    \item Veidot testus un uzglabāt atbildes uz tiem un to rezultātus;
    \item Testiem piešķirt variantu un uzdot studentam nejauši izvēlētus variantus izpildei;
    \item Ierobežot testu izpildes reizes un laikus;
    \item Importēt lietotāju datus no XLSX formāta dokumentiem;
    \item Eksportēt studentu sekmes XLSX formāta dokumentā;
    \item Izsūtīt lietotāja vārdu un paroli uz tam definēto e-pasta adresi.
\end{itemize}

\subsection{Dokumenta pārskats}

Dokumentu veido četri nodalījumi:

\begin{itemize}[noitemsep, nolistsep]
    \item Pirmajā nodalījumā – Ievads, iekļauta informācija par dokumenta vispārējo struktūru,
          nolūku un izmantotajām definīcijām;
    \item Otrajā nodalījumā – Risinājuma procesa apraksts, ir aprakstīti galvenie procesa soļi;
    \item Trešajā nodalījumā – Konkrētās prasības, aprakstītas visas risinājuma prasības, kas attiecas
          uz šo programmatūru;
    \item Ceturtajā nodalījumā – Risinājuma vispārējie ierobežojumi
\end{itemize}

\newpage