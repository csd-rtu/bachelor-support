\section{Ievads}

\subsection{Definīcijas un saīsinājumi}

\begin{tabular}{ |c|wc{12.5cm}| }
    \hline
    \rowcolor{blue!30}
    \textbf{Saīsinājums} & \textbf{Skaidrojums}                \\
    \hline
    RTU                  & Rīgas Tehniskā Universitāte         \\
    \hline
    BDIAS                & Bakalaura darbu IT atbalsta sistēma \\
    \hline
\end{tabular}

\subsection{Dokumenta nolūks}

Šis dokuments apraksta studiju priekšmeta "Datorsistēmu projektēšanas pamati" studija projekta "Bakalaura darbu IT atbalsta sistēma" (tālāk tekstā BDIAS) programmatūras prasības.

Dokuments ir paredzēts studiju projekta ietvaros programmatūras dizaina prototipa izstrādē, programmatūras ieviešanā un uzturēšanā iesaistītajām pusēm:

\begin{itemize}[noitemsep, nolistsep]
    \item pasūtītāja (RTU) mācībspēki, kuri atbildīgi par projekta nodevumu pieņemšanu un izvērtēšanu;
    \item izstrādātāja tehniskie speciālisti, kuri atbildīgi par tā realizāciju – projektēšanu un implementēšanu;
    \item studiju projekta dizaina prototipa izstrādātajiem.
\end{itemize}

\subsection{Darbības sfēra}

Studiju projekta ietvaros tiek izstrādāts sistēmas dizaina prototips, kas ļaus demonstrēt sistēmu, kur RTU studenti var:

\begin{itemize}[noitemsep, nolistsep]
    \item Saņemt informāciju par bakalaura darba izstrādi;
    \item Izvēlēties katedru, kur notiks bakalaura darba rakstīšana;
    \item Izvēlēties bakalaura darba tēmu no piedāvātā saraksta;
    \item Piedāvāt savu bakalaura darba tēmu;
    \item Saņemt paziņojumus par nepieciešamām bakalaura darba aktivitātēm;
    \item Augšupielādēt iesniegumus un bakalaura darbu uz pārbaudi;
    \item Saņemt atzīmes un komentārus par bakalaura darba izstrādes rezultātiem;
    \item Izmantot tērzētāvas funkcijas.
\end{itemize}

RTU bakalaura darba vadītāji, izmantojot BDIAS sistēmu, var:

\begin{itemize}[noitemsep, nolistsep]
    \item Augšupielādēt, pievienot un rediģēt bakalaura tēmas;
    \item Saņemt paziņojumus par nepieciešamām bakalaura darba aktivitātēm;
    \item Saņemt iesniegumus un bakalaura darbus uz pārbaudi;
    \item Likt atzīmes un rakstīt komentārus par bakalaura darba izstrādes rezultātiem;
    \item Izmantot tērzētāvas funkcijas.
\end{itemize}

RTU lietvedes, izmantojot BDIAS sistēmu, var:

\begin{itemize}[noitemsep, nolistsep]
    \item Saņemt iesniegumus no studentiem;
    \item Saņemt paziņojumus par saņemtajiem iesniegumiem;
    \item Veikt izmaiņas BDIAS sistēmā;
    \item Izmantot tērzētāvas funkcijas.
\end{itemize}

RTU programmas direktors, izmantojot BDIAS sistēmu, var:

\begin{itemize}[noitemsep, nolistsep]
    \item Apstiprināt iesniegumus no studentiem;
    \item Saņemt paziņojumus no sistēmas;
    \item Veikt izmaiņas BDIAS sistēmā;
    \item Izmantot tērzētāvas funkcijas.
\end{itemize}

RTU komisijas dalībnieki, izmantojot BDIAS sistēmu, var:

\begin{itemize}[noitemsep, nolistsep]
    \item Saņemt paziņojumus par aktivitātēm;
    \item Saņemt aizstāvēšanas datus;
    \item Izmantot tērzētāvas funkcijas.
\end{itemize}

\subsection{Dokumenta pārskats}

Dokumentu veido četri nodalījumi:

\begin{itemize}[noitemsep, nolistsep]
    \item Pirmajā nodalījumā – Ievads, iekļauta informācija par dokumenta vispārējo struktūru, nolūku un izmantotajām definīcijām;
    \item Otrajā nodalījumā – Risinājuma procesa apraksts, ir aprakstīti galvenie sistēmas procesa soļi;
    \item Trešajā nodalījumā – Funkcionālās prasības, aprakstītas visas risinājuma prasības, kas attiecas uz BDIAS;
    \item Ceturtajā nodalījumā – Nefunkcionālās prasības, aprakstītas visas risinājuma prasības, kas attiecas uz BDIAS.
\end{itemize}

\newpage